
\documentclass{beamer}

\mode<presentation>
{
 	\usetheme{PaloAlto}
 	\usecolortheme{beaver}	
 	\setbeamercovered{transparent}
 
}

\usepackage[english]{babel}

\usepackage[utf8]{inputenc}


\usepackage{times}
\usepackage[T1]{fontenc}


\title[Domain/OS] 
{Domain/OS}



\author[Manish Patel] 
{Manish Patel}


\institute[Loyola University in Chicago] % (optional, but mostly needed)
{
   Department of Computer Science\\
  Loyola University in Chicago
}
  

\date[12/14/12]


\subject{Operating Systems}
% This is only inserted into the PDF information catalog. Can be left
% out. 

% If you have a file called "university-logo-filename.xxx", where xxx
% is a graphic format that can be processed by latex or pdflatex,
% resp., then you can add a logo as follows:

% \pgfdeclareimage[height=0.5cm]{university-logo}{university-logo-filename}
% \logo{\pgfuseimage{university-logo}}


\begin{document}

\begin{frame}
  \titlepage
\end{frame}


\section{General Info}

\subsection{History}

\begin{frame}{History of Domain/OS}
  	\begin{itemize}
  	\item Developed by Apollo Computer Inc. in 1984
  	\item Predecessor to Aegis
  	\item Was primarly a network based OS
  	\item Bought by HP in 1988
  	\item No longer supported as of January 1st, 2001
  	\end{itemize}
 \end{frame}

\subsection{About}

\begin{frame}{About Domain/OS}
	\begin{itemize}
	\item Derived functionality from System V and BSD UNIX
	\item Ability to change enviroment variable with SYSTYPE command
	\item Developed in Pascal, but had compilers for C, C++, Pascal, and Fortran
	\item X Window System for windowing and handling the keyboard and mouse
	\item Visual User Enviroment for the desktop enviroment 
	\end{itemize}
\end{frame}

\subsection{Specs}

\begin{frame}{System Specs}
	\begin{itemize}
	\item 32-bit 20MHz MC68 Processor
	\item 4MB RAM
	\item 15 inch (1024x800) or 19 inch (1280x1024) color CRT
	\item 100MB, 200MB, or 400MB HDD
	\item 1.2MB 5.25 inch Floppy Drive
	\item 60MB .25 inch Catridge Tape Drive
	\item 2.3GB 8mm Tape Drive
	\item 12Mb/sec baseband Apollo Token Ring
	\item 10Mb/sec LAN Ethernet
	\item 4Mb/sec LAN IBM Token Ring
	\end{itemize}
\end{frame}

\section{Detailed Look}

\subsection{Kernel}

\begin{frame}{Kernel}
	\begin{itemize}
	\item Modified version of UNIX monolithic kernel
	\item Multiprocessor support
		\begin{itemize}
		\item UNIX was originally made for one processor
		\item Identify critical sections and how to protect them
		\item Synchronize all processes
		\item Mutual exclusive lock/unlock
		\item Process has to wait to aquire lock
		\end{itemize}
	\item Limited size of kernel
		\begin{itemize}
		\item Extends functionality with no modifications to kernel
		\item Moves services to libraries
		\item Avoids trapping kernel
		\end{itemize}
	\end{itemize} 
\end{frame}

\subsection{Libraries}

\begin{frame}{Dynamic Loading, Linking, and Sharing of Libraries}
	\begin{itemize}
	\item Classified as global and private
	\item Determined in System Configuration File
	\item Global
		\begin{itemize}
		\item Labeled as global and shared in configuration file
		\item Virtual addressing global spaces:
			\begin{itemize}
			\item Global Supervisor space hold limited kernel functions and data mapping
			\item Global User space hold libraries
			\item ie: close, open, read, and write
			\end{itemize}
		\item Entry points are stored in the Known Global Table
		\item Position-Indepent Code creates transfer vector
		\end{itemize}
	\item Private
		\begin{itemize}
		\item Labeled as dynamic or static in configuration file
		\item Dynamic: library loaded at program execution
		\item Static: library loaded when reference is made in program
		\end{itemize}
	\end{itemize}
\end{frame}

\subsection{Storage}

\begin{frame}{Single Level Storage}
	\begin{itemize}
	\item Multiple level storage have to explicitly copy data to primary storage to allow program to operate
	\item Domain/OS has one level
	\item Programs access objects by mapping object pages to process addresses
	\item Objects over the network are accessed sameway
	\item Demand Paging moves pages out of object dynamically over the network and locally
	\end{itemize}
\end{frame}

\section{References}

\begin{frame}{References}
	\begin{thebibliography}{3}
	
	\beamertemplatebookbibitems
	\bibitem{}
  	% Start with overview books.
   	 \newblock {\em Domain/OS Design Principles.}
   	 \newblock {\em Apollo Computer Inc., 1989.}
   	 
   	 \beamertemplatebookbibitems
   	 \bibitem{}
  	% Start with overview books.
   	 \newblock Domain/OS system Software Release Documentation
   	 
   	 \beamertemplatearticlebibitems
   	 \bibitem{}
   	 % Followed by interesting articles. Keep the list short. 
   	 \newblock http://web.mit.edu/kolya/www/csa-faq.html
   	 
   	 \end{thebibliography}
\end{frame}

\end{document}


